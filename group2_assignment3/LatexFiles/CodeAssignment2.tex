\documentclass{article}
\usepackage{geometry}
\usepackage[T1]{fontenc}
\geometry{letterpaper}

\usepackage{graphicx}
\usepackage{color}
\usepackage{listings}
\usepackage{amssymb}
\usepackage{epstopdf}
\DeclareGraphicsRule{.tif}{png}{.png}{`convert #1 `dirname #1`/`basename #1 .tif`.png}

\title{Brief Article}
\author{The Author}

\begin{document}
\begin{minipage}{\linewidth}
  \centering
 
\includegraphics[keepaspectratio=true, scale=0.35]{CMU.png}
  \section*{\emph{\bf Code Task 2 \hspace{0.2 cm} Due: 11:59 PM Mar 31\textsuperscript{st}, 2025\hspace{0.2cm}20 Points}}
  \section*{CSCI 250 Introduction to Algorithms}
  \section*{Graph Theory Algorithms}
  \section*{By: McKenzie Swindler, Stephen Bruner, Russell Cannon, Patrick Murphy\newline}

\end{minipage}


\section*{Task Description}
Construct a data input file containing the representation of the graph pictured below
a “DIGRAPH”. The “DIGRAPH” vertices labels are the integer in each node circle.
Arrows indicate edge directions. Edges without arrowheads are bidirectional and exist
as two directed edges.
Your program is to accept a single command line argument when invoked being the
data file name. Your program will then read the file contents into memory (using your
designed dynamic memory-based data storage structure). It would be unwise to code
your data structure based on the assumption that the graph size is limited to what is
pictured below. The test/demo graph may be larger (have extra nodes).
Once the program reads the file data, close it, you may not read from it again. Your
program will then prompt the user for three source vertices and destinations. 
Sample interaction:\newline\newline
Provide 3 sources: 7 19 26\newline
Destination 1?: 32\newline
Shortest Path 1 from vertex “Number”:\newline
Shortest Path 2 from vertex “Number”:\newline
Shortest Path 3 from vertex “Number”:\newline\newline
Destination 2?: 22\newline
Shortest Path 1 from vertex “Number”:\newline
Shortest Path 2 from vertex “Number”:\newline
Shortest Path 3 from vertex “Number”:\newline\newline
Destination 3?: 3\newline
Shortest Path 1 from vertex “Number”:\newline
Shortest Path 2 from vertex “Number”:\newline
Shortest Path 3 from vertex “Number”:\newline

\section*{Program Termination}
Upon a negative value user input the program will terminate, to do so it must:
\begin{itemize}
 \item Write a properly formatted log file of all interaction from invocation to termination
 \item For each Shortest path in the log it should also present running time of the calculation in nanoseconds
 \item De-allocate all dynamically allocated memory
 \item Record the overall runtime in seconds
 \item Record the date and time of the run
 \item Display a termination message on the screen
\end{itemize}

\section*{Our Program}
We decided to use the Breadth First algorithm to complete this task. Between BFS and DFS, BFS is best suitable for finding the shortest path between nodes.

\section*{Blacklist}
\begin{itemize}
  \item C++ code using STL constructs or libraries.
  \item Any data structure not of your making.
\end{itemize}

\textbf{Deliverables:}
\begin{itemize}
  \item C++ code implementing the Union-Find data structure and solving the problem.
  \item Classes should be coded using interface and implementations in separate les i.e., \.h" and\.cpp". Templated classes should be prototyped and implemented in their interface ".h" les. required).
  \item A LATEX based PDF document explaining your algorithm's time complexity analysis and any
optional features you implemented.
\end{itemize}

\vfill
\hrule

\begin{center}
{\scriptsize Last updated: \today\ - Typeset using: \textsc{t\kern -.12em\lower.4ex\hbox{e}\kern-.1em xs}hop by:\\ McKenzie Swindler \textcolor{blue}{Student of Computer Science} }
\end{center}

\end{document}
